\chap{Abstract} 


%%%% your abstract goes here (word limit: 350)
Many materials produced for studies in condensed matter physics are made using a limited set of "conventional" synthesis techniques and then characterized in a similar fashion. It is often assumed that samples prepared by different individuals, on different equipment, with different starting materials, and/or at different times are chemically identical provided their syntheses or diffraction patterns are similar. However, variations in other properties ranging from the subtle to the dramatic are commonly observed. More exhaustive characterization can reveal differences such as disorder, impurities, or changes in crystallographic orientation among samples, some of which can be the origin of these property changes. This is especially true for quantum materials, whose properties are contingent on the stabilization of a quantum mechanical state over a classical one which may be thermodynamically or entropically preferred under the vast majority of conditions.

Careful refinement of synthetic methods and a thorough approach towards characterization can help elucidate the properties of an “ideal” sample and guide materials refinement to produce desired properties. It may also expound upon differences between simulation and experiment, in turn providing useful information on how to improve theory tools.

This thesis opens with a discussion of how a material may be called "quantum", with further details on the concept of frustrated magnetism and the types of defects possible in these materials. In Chapters 2 and 3, I introduce the synthetic and characterization methods, respectively, used in the study of these materials. Chapter 4 discusses the sample variance of the frustrated magnet Yb2Ti2O7 and Chapter 5 details results of magnetism measurements on single crystals of this material as compared to simulations by a quantum annealer. Chapter 6 covers a study of the perovskite as a potential qubit host, including a discussion on material aging in this system. Chapter 7 concludes with a description of a metric developed for quantification of unit cell shape in that material, which has also been extended to other materials families.


%% list of keywords seperated by comma
\keywords{Johns Hopkins, PhD, dissertation, thesis, \LaTeX, template.}


%%%%  committee members (add it right after the abstract w/o page break)
\begin{singlespace}

    %% if you have co-advisor, add here w/ \vspace{0.1in} as shown below
    %% alternatively you can use minipage environment to put side-by-side
    \section*{Primary reader and thesis advisor}
    
    Dr. Tyrel M. McQueen \\
    Professor\\
    Departments of Chemistry, Physics, and Materials Science\\
    Johns Hopkins University, Baltimore MD 


    \section*{Secondary readers}
    
    Dr. Placeholder Hawking\\
    Professor\\
    Department of Departments \\
    Johns Hopkins University, Baltimore, MD 
    
    \vspace{0.1in}
    
    Dr. Placeholder Turing \\
    Professor\\
    Department of Paperwork \\
    Johns Hopkins University, Baltimore, MD 

    %% you can add more readers if you have them on your committee 
    %% use \vspace{0.1in} in between members for clarity
    %% you can also place committee members side-by-side using `minipage`


\end{singlespace}