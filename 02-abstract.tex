\chap{Abstract} 


%%%% your abstract goes here (word limit: 350)
Quantum materials are cool.

This thesis opens with a discussion of how a material may be called "quantum", with further details on the concept of frustrated magnetism and the types of defects possible in these materials. In Chapters 2 and 3, I introduce the synthetic and characterization methods, respectively, used in the study of these materials. Chapter 4 discusses the history of the frustrated magnet Yb2Ti2O7 and Chapter 5 details results of magnetism measurements on single crystals of this material. Chapter 6 covers a study of the perovskite Sr2CaWO6 for potential quantum sensing applications, including a discussion on aging in this system. Chapter 7 concludes with a description of a metric developed for quantification of unit cell shape in that material, which has also been extended to other materials families.


%% list of keywords seperated by comma
\keywords{Johns Hopkins, PhD, Masters, dissertation, thesis \LaTeX, template.}


%%%%  committee members (add it right after the abstract w/o page break)
\begin{singlespace}

    %% if you have co-advisor, add here w/ \vspace{0.1in} as shown below
    %% alternatively you can use minipage environment to put side-by-side
    \section*{Primary reader and thesis advisor}
    
    Dr. Tyrel M. McQueen \\
    Professor\\
    Departments of Chemistry, Physics, & Materials Science\\
    Johns Hopkins University, Baltimore MD 


    \section*{Secondary readers}
    
    Dr. Placeholder Hawking\\
    Professor\\
    Department of Departments \\
    Johns Hopkins University, Baltimore, MD 
    
    \vspace{0.1in}
    
    Dr. Placeholder Turing \\
    Professor\\
    Department of Paperwork \\
    Johns Hopkins University, Baltimore, MD 

    %% you can add more readers if you have them on your committee 
    %% use \vspace{0.1in} in between members for clarity
    %% you can also place committee members side-by-side using `minipage`


\end{singlespace}