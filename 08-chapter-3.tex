\chapter{Characterization methods} \label{chap:chap-1}

% if you want a short header you can use the following command
% \chapter[short-header-name]{chapter-title} \label{chap:chap-1}


%% add your chapter text here
\begin{singlespace}         % you can also use `onehalfspace` to relax the spacing
    Portions of this chapter are adapted from a manual written for the McQueen Lab/IQM Laue diffractometer setup with assistance from Alireza Ghasemi, Wyatt Bunstine, and Allana Iwanicki.

\end{singlespace} 
\section{X-ray diffraction}
\subsection{Diffraction basics}
\begin{figure}[htbp]
  \centering
  \includesvg[width = 200pt]{Basic wave.svg}
  \caption{A sinusoidal wave}
  \label{basicwave}
\end{figure}
Diffraction is a term used to describe interference of a large number of waves. The principle of superposition of waves states that two waves incident on the same point will interfere, resulting in a new wave with amplitude equal to the vector sum of the amplitudes of the initial waves. One can describe the frequency f of photons (light) with any basic sinusoidal equation, such as Equation \ref{PhotonFreqSinusoidal}:
\begin{equation}
     f(t,\vec{x})=A\cos{(\omega t-\vec{s}\cdot\vec{x}+\varphi)}
    \label{PhotonFreqSinusoidal}
\end{equation}
where t is time, $\vec{x}$ is a vector representing position, and A, $\omega$, $\vec{s}$, and $\varphi$ are constants. Waves described in such a way and traveling along the same path $\vec{x}$ will constructively interfere (sum intensity) when they have the same phase $\varphi$ and deconstructively interfere (cancel intensity) when their phases are \ang{90} ($\pi$) apart, as illustrated in Figure \ref{interference}.
\begin{figure}[htbp]
  \centering
  \includesvg[width = 400pt]{Interference.svg}
  \caption{Constructive (left) and deconstructive (right) interference}
  \label{interference}
\end{figure}
Diffraction is possible with all types and frequencies of waves, although waves will only diffract when they are incident on an object which enforces the necessary conditions for diffraction. One such object is a diffraction grating, which has slits (at least as wide as the wavelength) spaced to restrict light waves to follow a certain set of paths $\vec{x}$. For a given wavelength $\lambda$ for two waves traveling on the same path, a phase difference $\Delta\varphi$ can be produced when the two waves travel different distances d, and this can enable diffraction.

In the case of x-ray diffraction, or XRD, the waves in question have a frequency between $3\times10^{16}$ Hz to $3\times10^{19}$ Hz (about 10 nm to 10 pm), termed "x-rays". This wavelength range corresponds to the spacing of atoms in a crystal.


!!!In a back reflection setup FIGURE, constructive interference of light incident on a 1D plane of atoms with lattice spacing d is subject to equation \ref{Bragg'sLaw} (Bragg’s law defined normal to the plane).!!! Only x-rays in the reflected beam will reach the detector. If n is not an integer, x-rays will deconstructively interfere and be destroyed before they can reach the detector.
\begin{equation}
    2d\cos{\theta} = n\lambda
    \label{Bragg'sLaw}
\end{equation}
Equation \ref{latticeplanespacinggeneral} defines the lattice plane spacing d in terms of the plane number h k l and lattice constants a, b, c, $\alpha, \beta, \gamma$ for a triclinic crystal. It may look like a nightmare, but it reduces for higher symmetry space groups, all the way to the very simple equation \ref{latticeplanespacingcubic} for a cubic crystal. Equation \ref{latticeplanespacinghexagonal} defines d for a hexagonal crystal.

\begin{multline}
    \frac{1}{d^2}=\frac{h^2b^2c^2\sin^2{\alpha}+k^2a^2c^2\sin^2{\beta}+l^2a^2b^2\sin^2{\gamma}}{a^2b^2c^2(1-\cos^2{\alpha}-\cos^2{\beta}-\cos^2{\gamma}+2\cos{\alpha}\cos{\beta}\cos{\gamma})}\\\times[2hkabc^2(\cos{\alpha}\cos{\beta}-\cos{\gamma})+2kla^2bc(\cos{\beta}\cos{\gamma}-\cos{\alpha})\\+2hlab^2c(\cos{\alpha}\cos{\gamma}-\cos{\beta})]
    \label{latticeplanespacinggeneral}
    \end{multline}
\begin{equation}
    \frac{1}{d^2}=\frac{h^2+k^2+l^2}{a^2}
    \label{latticeplanespacingcubic}
\end{equation}
\begin{equation}
    \frac{1}{d^2}=\frac{4}{3}\frac{h^2+hk^2+k^2}{a^2}+(\frac{l}{c})^2
    \label{latticeplanespacinghexagonal}
\end{equation}
The smallest possible d spacing which can be resolved corresponds to the smallest wavelength\footnote{The existence of a minimum and maximum in both the emitted light AND in what will diffract is why interference is limited and we are able to see a pattern of spots.}:
\begin{equation}
    d_{min}=\frac{\lambda_{min}}{2}
    \label{minspacing}
\end{equation}
\subsection{Information extracted from a Laue measurement}
\begin{singlespace}
During regular use:
\begin{itemize}
    \item A measure of crystallinity and disorder from number, symmetry, and shape of spots
    \item Info on domains and twinning from spot symmetry
    \item Precise crystal orientation from spot locations and intensities. Non-crystalline samples may show orientation preference.
    \item The Laue class and crystal type from primarily spot symmetry 
    \item Crystallochemical info, \textit{i.e.} a rough check if you have the correct sample, from the Laue class and spot spacing
\end{itemize}
\end{singlespace}
During specialized use:
\begin{itemize}
    \item A rough approximation of lattice parameters if tube voltage (\textit{i.e.} minimum $\lambda$) and sample-film distance are known
    \item A measure of beam divergence by following careful analysis of spot shapes on a very highly-crystalline sample 
    \item Thermal fluctuation information from diffuse scattering, if present. Typically a measurement looking for this specifically would use filters to reduce all but characteristic tube radiation (and probably a non-W source) and vary the temperature.
    \item Time-resolved crystallography using a very specialized setup (not doable by us, but at a synchrotron) as in https://www.ncbi.nlm.nih.gov/pmc/articles/PMC6501890/ 
\end{itemize}
\subection{The detector}
Our detector is a 30 x 30 cm wide screen, behind which is a chamber filled with gas and wire mesh. X-rays ionize Ar gas particles (10\% CO\textsubscript{2} limits chain reactions where one x-ray leads to several ions), which are drawn to the wire mesh held at a negative voltage. This is called a multi-wire proportional counter.

\subection{The x-ray source}
X-rays are produced in an x-ray lamp by using high voltages to accelerate electrons onto an anode FIGURE, producing x-rays and heat. The heat is why water cooling is essential! X-rays are produced at a broad range of wavelengths (called bremsstrahlung radiation) as well as a few specific wavelengths characteristic of the specific electronic transitions in the anode material.
\begin{figure}[htbp]
  \centering
  \includesvg[width = 400pt]{unfiltered tungsten radiation.svg}
  \caption{Tungsten x-ray lamp radiation fluence at different biases (SMB-made with data from XXX.}
  \label{Wradiation}
\end{figure}
Figure \ref{Wradiation} shows the measured intensity vs. wavelength for an unfiltered tungsten anode (used here) when different biases are applied.\footnote{Intensity data from: https://oar.ptb.de/resources/show/10.7795/720.20201118 with line positions from: https://xdb.lbl.gov/Section1/Table_1-2.pdf }  Note that the intensities here are unfiltered, but the 300 $\mu$m beryllium window used in our setup will uniformly cut intensity across all wavelengths. Intensity is arbitrary anyway, so you can treat this graph as accurate for our setup.
\begin{equation}
    \lambda_{min} = \frac{hc}{eV_{bias} }
    \label{lambdamin}
\end{equation}
Equation \ref{lambdamin} shows the minimum possible wavelength $\lambda$ accessible with a given anode bias Vbias (h = Planck’s constant, c = speed of light, e = electron charge). In Laue diffraction, we use white radiation (aka polychromatic, or many wavelengths), so we don’t use filters like a nickel filter to remove the background.\footnote{Note that the lack of a filter increases the danger to human tissue, which does interact with low-energy x-rays (unlike crystals).} Instead, we work below 50 kV bias, where the very intense K peaks for tungsten ($\lambda~0.2 $ \AA) will not be seen. At the normal operating condition of 10 kV, the radiation is entirely broad bremsstrahlung background – this is good for Laue diffraction. The most important part of the spectrum is the part around 1 \AA – on par with the size of the spacing between atoms that we would like to characterize - which is optimized between 10 and 15 kV.
Note that if you are using high enough voltages to excite the characteristic emissions of the tungsten anode, you will see strangely intense spots from these characteristic x-rays! These are called “Bragg” spots, while the rest are “Laue” spots. The larger the unit cell, the more likely you are to see Bragg spots [when you have sufficient tube voltage].
It is possible to use characteristic intensities to study the diffuse scattering of a crystal. This diffuse scattering increases with temperature because it is due to thermal motion. It also has the interesting property of remaining in the same place no matter how you rotate your crystal.
Laue diffraction comes in two geometries: transmission and back-reflection. We use back-reflection. The relatively low intensity of x-rays here would limit the sample thickness if we were using transmission because, in order to reach the detector, an x-ray would have to pass all the way through the sample. In back-reflection, this is less of a concern, although highly absorbing samples will produce fainter patterns.

\section{Spot intensities}
In Laue diffraction, a stationary crystal is bombarded with many wavelengths. This is in contrast with powder or single crystal diffraction, wherein a rotating crystal or powder is bombarded with one or two wavelengths. Since d and θ are fixed, all planes that make the same angle θ with the source will be superimposed on one Laue “spot” FIGURE
You can think of each plane as “choosing” the right λ for its diffraction condition out of the many available. The intensity of a Laue spot is the sum of all the diffracting x-rays with the same scattering angle θ that reach the detector.
This means that nominally “forbidden” reflections may still produce spots.
It also means that lower index numbers will have the most intense spots, because the lower the index number, the more symmetrically equivalent planes there are to reflect x-rays.
{110} = (110) + (220) + (330) + (440) +…
\subsection{Thomson scattering}
The intensities of x-rays coming from individual planes are mostly determined by equation \ref{planarintensity} for Thomson scattering FIGURE off an electron, in which an x-ray colliding with an electron excites it (into higher oscillatory motion), and then the electron emits an x-ray to relax to a lower energy state. There’s a phase change (which ends up not mattering for most situations) but otherwise no difference between the incoming and outgoing electrons. 
\begin{equation}
    I_{scat}(\theta,\lambda)=I_{beam}(\lambda)\frac{e^4}{m^4c^4R^2}(\frac{1+\cos^2{2\theta}}{2})\times|F_{hkl}|^2
    \label{planarintensity}
\end{equation}
\begin{equation}
    F_{hkl}= \sum_{j}^{\text{atoms in unit cell}} f_je^{2\pi i(hx_j+ky_j+lz_j)}
    \label{Fhkl}
\end{equation}
Here, R is the distance from the electron to the place where you measure intensity I, m is the electron mass, and e is the electron charge. Fhkl is the structure factor, depending on the position xj yj zj of atom j and its atomic scattering form factor fj. Tabulated fj roughly describe the efficiency of an element’s x-ray scattering, similar to an absorption factor.\footnote{fj decreases as sinθ/λ increases, so back-scattering has worse contrast than transmission Laue.} Fhkl can be computed from cifs. Equation \ref{planarintensity} does not include corrections for beam divergence, absorption, nor geometrical considerations but is proportional to the complete result.\footnote{See Preuss et. al. for an easy to follow discussion of these corrections.}
When Fhkl is 0, the reflection is “forbidden”. This doesn’t mean the spot won’t show up in Laue, however. If, for example, the (120) reflection is forbidden, you might still get a spot from  Σ(240) + (480) + …. Although (120) doesn’t show up, there is a spot at its location. More on this in a later section.
X-rays primarily interact with the cloud of electrons surrounding an atom, which explains why atomic radius is more measurable in x-ray diffraction than nucleus size. Thomson scattering increases with atomic number because there are more electrons.
\subsection{Compton scattering}
An additional mechanism, called Compton scattering FIGURE, occurs when an incoming x-ray acts like a particle and physically displaces an electron, like two balls colliding. This only really happens when the electron being smacked around isn’t held tightly by its parent atom. The photon goes off at a new angle and has lost some energy during this elastic collision. It’s also picked up some phase that has nothing to do with its original phase. This random phase means it won’t diffract according to the known conditions, but it will increase the background.\footnote{Compton scattering is inelastic, and ∆λ∝1-cos⁡2θ so in back-scattering where 45°≤θ≤90° Compton scattering is worse than in transmission Laue.}

The looser held the electrons, the more Compton scattering you get – in practice this means Compton scattering decreases with increasing atomic number. Lower atomic number elements produce lots of background. As a result, organics are not great for Laue.
\subsection{Crystal perfection}
Counter-intuitively, absolutely perfect crystals produce weak patterns. The reason for this is that a phenomenon called primary extinction reduces the intensity of the incident beam as it travels through the crystal, so deeper planes receive fewer x-rays. Primary extinction is caused by scattering in one plane – e.g. (200) – redirecting an x-ray antiparallel to its original trajectory with a 90° phase shift. When this x-ray interacts with a higher plane – e.g. (100) – it scatters and gets another 90° phase. The final beam is now parallel to the initial with the exact 180° phase difference needed for deconstructive interference Fig 31.
Real materials actually have a kind of mosaic structure FIGURE with lots of micro-domains oriented similarly. This allows tiny variations in the phases of diffracted x-rays which prevent completely destructive interference.
Even in mosaic crystals, outer planes will always experience greater intensity than inner ones. If the inner planes are at the same orientation as the outer ones, they receive much less because the outer planes have already diffracted (i.e. redirected) the x-rays out of the material. Laue is semi-surface sensitive.
Individual defects in a crystal act like lone atoms, scattering x-rays in all directions. This contributes to a broad background and reduces the intensity of spots by comparison. Completely amorphous samples have basically no ordered interference and no spots. You may be able to see some broad but weak variations in intensities if there is a preferred orientation.
\subsection{Summary of factors influencing intensity}
\begin{itemize}
    \item Index number – lower index means more intense, in general
    \item Absorption factor – less absorbing element = more intense
    \begin{itemize}
        \item Absorption can also create double spots with unusual intensity, but this is rare.
        \end{itemize}
    \item Number of equivalent reflections summing to produce one spot – more = darker
    \item Interference effects – constructive or deconstructive based on unit cell and λ
    \item Atomic number – higher Z leads to more Thomson and less Compton scattering
    \item Amount of material in the beam – larger crystals have more stuff literally that diffracts
    \begin{itemize}
        \item This also applies to larger domains within a multi-domain crystal
        \end{itemize}
    \item Size of collimator – the larger the collimator, the more x-rays make it to the sample
    \item Detector distance – more reflected x-rays hit the detector when the sample is closer
    \item Beam divergence – the Lorentz correction factor for white radiation ∝ 1/sin2θ
\end{itemize}
Lower tube voltage does in general reduce x-ray beam intensity, but the number of x-rays diffracted back to the detector is rarely affected.
Some more rare effects to be aware of (though usually not important):
\begin{itemize}
    \item Temperature – thermal oscillations reduce spot intensity (they do not broaden spots)
    \item Crystal perfection – there is a sweet spot between disorder and perfection for max I
    \item Surface reflectivity – an x-ray-reflective surface reduces overall spot intensity
    \item Characteristic x-rays (if above 15 kV) – these will create anomalously intense spots
    \item Fluorescence – some elements may fluoresce (i.e. emit new, non-diffracted electrons) when exposed to radiation, which results in anomalously intense spots
    \begin{itemize}
        \item The main way to reduce this on our setup is to use lower tube voltage. See http://www.lauecamera.com/index-Dateien/Page787.htm for additional ideas.
        \end{itemize}
\end{itemize}
\section{Sizes and shapes of spots}
\subsection{"spots" vs. "lines"}
Since we have a 2D image in Laue diffraction, we are expecting to see polygons rather than lines FIGURE. The shape of these polygons is determined by a variety of factors including beam shape and sample symmetry. See, for example, how 2D lines of molecules diffract into lines, but 3D shapes diffract into spots in these simulated patterns.\footnote{An excellent visual discussion of diffraction is available here: https://www.xtal.iqfr.csic.es/Cristalografia/index-en.html } Note that these are not Laue patterns and were produced in a different geometry and with monochromatic radiation, but the lines vs. spots idea is still true.
We are looking at 3D crystals, so we expect diffracted spots. Given that the cross-section of the x-ray beam leaving the collimator is circular, we also generically expect circularly-symmetric spots. These will be centered exactly at the points predicted by our diffraction conditions. They should be small since there are many opportunities for destructive interference.
\subsection{Shape variation}
d-spacing is large relative to the possible orientation variation in a decent crystal. We thus know that different planes diffract x-rays to vastly different locations (i.e. different spots)…whereas tiny variations in planar orientation will send x-rays to mostly the same place. Just like peaks in powder diffraction are lines broadened by the presence of varied crystal or domain orientations, our points are broadened, about equally in x and y directions.
You may, however, notice that the points towards the edge of your pattern are increasingly elliptical but not circular. The reason for this is schematically illustrated in FIGURE (for the transmission geometry), and results from the fact that the x-ray beam is divergent – the x-ray source produces x-rays in all directions radiating outward, not in just one parallel direction. 
When an x-ray beam is encountering a material with fixed orientation, but the individual incident photons have slightly varying directions, the diffracting photons will also have varying directions. In fact, they will vary linearly with the distance along the crystal (between A and B in the diagram). Thus a circular incident beam becomes linearly “squished” into an ellipse. When the x-rays of interest are moving normal to the plane (i.e., for the spots from the sample’s current orientation), their angular variation is much smaller and the spots come out more circular. Circular spots near the center tell you which axis you are “looking down”.
The discussion above also tells us that a broader spot in general results from greater variation in θ but is independent of variation in d-spacing. Smaller spots indicate higher crystallinity by physically showing that x-rays are diffracting to the same narrow area on the detector. Spots may be distorted from their true elliptical shapes by the presence of multiple, significantly different domains producing literally overlapping patterns. We have several examples of this in our Example Data section above, which illustrate that Laue patterns can be a quick way to check crystal quality. However, note that some spot broadening could be caused by the way the incident x-ray beam diverges – and that it might not be symmetrical if the beam isn’t symmetrical or if the spot is made of multiple reflections which behave differently!
You can cut down on beam divergence by using a smaller collimator, and this is half of the reason why smaller collimators produce higher resolution patterns with smaller, better separated spots. The other half of the reason is lower incident beam intensity resulting in less background, as discussed above.
Therefore, if you suspect some of your spot broadening or lack of circular symmetry may be due to poor crystal quality, you should swap to a smaller collimator and increase your exposure time. If the spots begin to look more like figure 8’s than ellipses, you have a second domain. Clovers would indicate three domains. Both shapes are highly unlikely to be from beam divergence. 
\section{Spot positions}
\subsection{The reciprocal lattice}
The reciprocal lattice is a description of what an x-ray will interact with. It has the useful feature of sharing the symmetry of its parent lattice, and it is what we actually image in Laue diffraction (rather than the actual real-space location of your atoms). The precise definition of the reciprocal lattice is the set of all vectors that are wavevectors of plane waves in the Fourier series of a spatial function with the periodicity your real [direct] lattice.
To better understand this concept, note that plane waves can be written as functions of wavenumber k and position x (see equation 14 for the precise form). The direct [real] lattice exists in Euclidean space (the real world), where length is measured by x, FIGURE. The reciprocal lattice exists in k-space, where length is measured by k,FIGURE. They both describe the same thing, but the reciprocal lattice is more useful for talking about diffraction because a point on the reciprocal lattice corresponds to a plane wave in real space whose phase is 2πn at every direct lattice point. Recall that 2πn is the constructive interference condition. Thus the reciprocal lattice has points everywhere incident x-rays should constructively interfere!\footnote{See also https://www.youtube.com/watch?v=DFFU39A3fPY for a visualization} The same can not be said of the direct lattice.
Just like real crystals are described by unit vectors a b c , the reciprocal lattice is described by unit vectors a* b* c*, which can be calculated from the unit vectors of your initial real-space lattice. K is a constant relating to the volume of your initial lattice.

\begin{equation}
    \vec{a^*}=K\frac{\vec{b}\times\vec{c}}{\vec{a}\cdot(\vec{b}\times\vec{c})};\quad
    \vec{b^*}=K\frac{\vec{c}\times\vec{a}}{\vec{b}\cdot(\vec{c}\times\vec{a})};\quad
    \vec{c^*}=K\frac{\vec{a}\times\vec{b}}{\vec{c}\cdot(\vec{a}\times\vec{b})}
    \label{RecipLatVecs}
\end{equation}

The reciprocal lattice is the set of all reciprocal lattice vectors x*. The following relationship holds between the real space lattice vectors x and the reciprocal lattice vector x*:
\begin{equation}
    \vec{x^*}=h\vec{a^*}+k\vec{b^*}+l\vec{c^*}
    \label{RecipLatVecX}
\end{equation}
\begin{equation}
    \vec{x}=p\vec{a}+q\vec{b}+r\vec{c}
    \label{RealLatVecX}
\end{equation}
\begin{equation}
    \vec{x}\cdot\vec{x^*}=2\pi n
    \label{RecipRealLatVecXRelation}
\end{equation}
Points in the reciprocal lattice – the points imaged by Laue diffraction – are determined by the location and spacing of planes in the real crystal. Thus each point corresponds to a different plane. The reciprocal lattice is insensitive to individual variations in atom types and positions, but it is sensitive to variations in d-spacing or the angles between planes.
\subsection{The Ewald sphere and the Laue condition}
The positions of Laue spots are ultimately determined by the Ewald sphere FIGURE, which is a kind of extension of Bragg’s law into three dimensions, now including the idea of the reciprocal lattice. To be precise, the Ewald sphere (also more intuitively called the “sphere of reflection”) is a sphere of radius 1/λ passing through the origin of the reciprocal lattice. 
Imagine a diffracting x-ray from a single plane of a crystal. In the reciprocal lattice, it interacts with a single reciprocal point. It will be sent in a specific direction out of the crystal determined by the diffraction conditions and the incident direction. If you rotated the crystal, the same incident x-ray would be sent to a new point, but it’s always interacting with the same reciprocal lattice and subject to the same diffraction conditions; namely, that the wavelength of the incoming and outgoing x-rays be the same (elastic scattering) even though their directions vary.
It turns out that Bragg’s law can be rewritten in terms of the directions of the x-rays, instead of in terms of nλ, and still describe their interference. First, we define the scattering vector Δs to describe the difference between the incoming and outgoing photon momenta with vectors s.
\begin{equation}
    \vec{s}_{diffracted}-\vec{s}_{incident}=\Delta\vec{s}
    \label{ScattVec}
\end{equation}
 Since we have elastic scattering, we know frequency doesn’t change between the incident and outgoing photons. Thus we can also write equation \ref{FreqInFreqOut1} to relate fin to fout. Equation \ref{FreqInFreqOut1} holds true whenever the phase varies by 2πn because cosine is a periodic function (this same relationship is what gave Bragg’s law, equation 2). This gives equation \ref{FreqInFreqOut2}. 
\begin{equation}
   \cos{(\omega t-\vec{s}_{in}\cdot\vec{x}+\varphi_{in})}=\cos{(\omega t-\vec{s}_{out}\cdot\vec{x}+\varphi_{out})}
    \label{FreqInFreqOut1}
\end{equation}
\begin{equation}
    \Rightarrow\omega t-\vec{s}_{in}\cdot\vec{x}+\varphi_{in}=\omega t-\vec{s}_{out}\cdot\vec{x}+\varphi_{out}+2\pi n
    \label{FreqInFreqOut2}
\end{equation}
\begin{equation}
    \Delta\vec{s}=\vec{x^*}
    \label{FreqInFreqOut3}
\end{equation}
Combining equations \ref{ScattVec} and \ref{FreqInFreqOut1}, you can get equation \ref{FreqInFreqOut3} with some rearranging … and this looks awfully similar to equation \ref{RecipRealLatVecXRelation} for the relation between reciprocal and real lattice vectors. Thus we are able to finally derive equation \ref{LaueCondition} called the Laue condition for diffraction (after Max von Laue, not the specific technique) below: 
\begin{equation}
    \Rightarrow\Delta\vec{s}\cdot\vec{x}=2\pi n
    \label{LaueCondition}
\end{equation}
The key to the Laue condition (which was developed before Bragg’s law)  is that the lengths of both s vectors are equal to 1/λ – and λ is the wavelength of both the diffracted and incident beams. If you draw sin and sout with the same origin, their endpoints lie on a circle of radius 1/λ FIGURE. The difference between these is the third leg of an isosceles triangle between the origin and two points on the edge. Stated equivalently, any reciprocal lattice vector that connects two points on the circle meets the diffraction conditions. 
Include the third dimension, and you have the Ewald sphere!
What we “see” in Laue diffraction is the inside surface of a conglomerate Ewald sphere made up of all the Ewald spheres for all the wavelengths of light in the incident beam. This is what QLaue simulates, for example.\footnote{To visualize a single Ewald sphere without polychromatic radiation, you can try this Java applet: https://www.xtal.iqfr.csic.es/Cristalografia/Applet/Ewald/diffractOgram.jar}
This seems counterintuitive: when we see “spots” in the diffraction patterns, we may assume that we are looking at the exact positions of atoms (see bottom row of FIGURE for a good illustration). However, this is not the case in Laue diffraction. To see the exact locations of atoms, we would use a different technique, such as electron microscopy.
Note that since there is also deconstructive interference, we are able to resolve distinct spots even from this conglomerate Ewald sphere. 
\subsection{Laue zones and spot symmetry}
It is time now to correct an imprecision in our discussion of Ewald spheres which is: we don’t actually see the whole Ewald sphere in Laue diffraction. Instead, we see the intersection of the sphere with our flat detector FIGURE .\footnote{A cylindrical detector may also be used; see Amorós et. al. for Laue patterns in different geometries.} And, as only a fixed set of angles (between 45° and 90°) can reach our detector, what we are truly looking at is the intersection of a cone with a vertical plane FIGURE. The field of math describing this intersection is called conics; the key visuals are:
\begin{enumerate}
    \item We image zones of points which all come from planes that share a common axis.
    \item A zone appears like a straight line when the zone axis is perpendicular to the x-ray beam (when we are “looking down” the axis).
    \item A zone appears like a hyperbola\footnote{In transmission Laue, zones may also be elliptical or parabolic in addition to straight and hyperbolic.} when the zone axis is not perpendicular.
    \item The zone curvature increases as you get further away from perpendicular.
\end{enumerate}
There are as many zones as there are axes of symmetry in your crystal. You also know from the symmetry of the reciprocal lattice that an axis with, say, 4-fold symmetry will also have 4-fold symmetric zones. The symmetry of a spot indicates the symmetry of the corresponding axis. All the spots in a zone belong to a single type of reflection, and are called first-, second-, third-order etc. based on whether they are the n=1, n=2, n=3… in Bragg’s law.
The symmetry of your crystal also influences whether you will have zones at all. Spots are created only when their originating axis meets the reflection conditions for a given space group.\footnote{You can see the reflection conditions for more complicated space groups here: https://www.cryst.ehu.es/cryst/get_hkl.html } These are due in general to the Wycoff positions of your atoms, and there will be a set of conditions for each center of symmetry, glide plane, and screw axis. In a face-centered cubic (FCC) crystal FIGURE, for example, reflections only occur when h k l are all odd or all even. Even though second- or higher-order reflections may contribute some extra points (because they interact strongly with some other λ in the white radiation), there is still a lot less resulting zone symmetry when compared to a body-centered cubic (BCC) pattern FIGURE, which only requires even Σh + k + l.
\subsection{Limitations of symmetry - the Laue groups and Friedel’s law}
A Laue pattern is always centrosymmetric even if the crystal structure generating it has no center of symmetry. This means that, out of the 32 point groups possible, we see only 11 distinct Laue groups (sometimes called “classes”), as in Table \ref{LaueGroups}.
\begin{table}[]
\caption{The Laue Groups}
\begin{tabular}{lllll}
\textbf{Crystal   system}   & \textbf{Point group}  & \textbf{Laue group} &  &  \\
Triclinic                   & 1, -1                 & -1                  &  &  \\
Monoclinic                  & 2, m, 2/m             & 2/m                 &  &  \\
Orthorhombic                & 222, mm2, mmm         & \textit{mmm}        &  &  \\
\multirow{2}{*}{Tetragonal} & 4, -4, 4/m,           & 4/m                 &  &  \\
                            & 422, 4mm, -42m, 4/mmm & \textit{4/mmm}      &  &  \\
\multirow{2}{*}{Trigonal}   & 3, -3,                & -3                  &  &  \\
                            & 32, 3m, -3m           & -3m                 &  &  \\
\multirow{2}{*}{Hexagonal}  & 6, -6, 6/m,           & 6/m                 &  &  \\
                            & 622, 6mm, -6m2, 6/mmm & 6/mmm               &  &  \\
\multirow{2}{*}{Cubic}      & 23, m3,               & \textit{m3}         &  &  \\
                            & 432, -43m, m3m        & \textit{m3m}        &  & 
\end{tabular}
\label{LaueGroups}
\end{table}
Friedel's law states that the magnitudes of the h, k, l and -h,-k,-l reflections are equal (there is a phase difference, but we can’t detect it) whenever the crystal is centrosymmetric and/or there is no resonant scattering. Resonant scattering occurs whenever certain wavelengths scatter better off certain elements and you need to describe the scattering factor as follows:
\begin{equation}
    f=f_o+f'+if''
    \label{Friedel'sLaw}
\end{equation}
f'and f’’ are very small except near absorption edges of the element, which are locations where the absorption spectrum changes abruptly because an electronic transition is present at the same energy of the absorbing photon. You may also hear the terms anomalous scattering and anomalous dispersion used to describe the same wavelength-dependence.
When absorption is negligible and Friedel's law applies, it’s impossible to distinguish between a centrosymmetric point group and one of its non-centrosymmetric subgroups; in this case only point groups belonging to different Laue classes can be distinguished. Thus, even though zone symmetry is a great tool for identifying lattice types in general, it can’t determine the space group (nor lattice parameters, since we’re using polychromatic radiation).\footnote{See Amorós et. al. for a detailed discussion of why this is true.}
\subsection{Spot (x,y) locations on detector}
It is possible to compute the specific locations of each spot (separate from their zones and expected symmetry, see FIGURE). The equations depend on the space group, but as an example for the (100) axis of a cubic crystal 
\begin{equation}
    (y,z)
    \label{CubicSpotYZ}
\end{equation}
\begin{equation}
    (\theta,\phi)
    \label{CubicSpotAng}
\end{equation}
\subsection{Stereographic projections}
The stereographic projection is a representation of the expected spot pattern from a 3D crystal on a 2D detector. If you picture a crystal inside a sphere and draw a line normal to each face out to the edge of the sphere AFIGURE, you will have a sphere with points on its surface at the intersection of each line. If you then trace those points back down onto a flat plane, you will have a 2D representation called a stereographic projection. In general, a stereographic projection is a 2D projection of a sphere through a surface point onto a plane perpendicular to its diameter through the point. As a conformal map, it also preserves the shapes of objects on the sphere in the flat projection as well as the angles between them – circles remain circles, scalene triangles remain scalene, etc. In crystallography, the stereographic projection reflects the symmetry of the originating space group and will vary depending on the axis you are “looking down”.
A pole figure is a graphical representation of the orientation of an object in space. If you take a stereographic projection of the reciprocal lattice of your space group, then label each point with an axis, you have created a pole figure. This process is called indexing;\footnote{You can also make pole figures with other projections, such as the gnomonic projection (image) which includes multiple index reflections on the same Laue points. See Amorós et. al. for a list of the options and their benefits.} see above for information on how to do it in practice.